\documentclass[xcolor=table]{beamer}
%\usepackage{etex} % Fixes out-of-registers error. Probably some bad package.

\usepackage[utf8x]{inputenc}
%\usepackage{default}
\usepackage[pdftex]{epsfig}
\usepackage[table]{xcolor}

\usepackage{tikz}
\usetikzlibrary{calc}

\usepackage{pdfpages} % Include other pdf files as slides

\setbeamertemplate{navigation symbols}{}

\renewcommand\sfdefault{phv}
\renewcommand\familydefault{\sfdefault}
\usetheme{default}
\usepackage{color}
\useoutertheme{default}

\setbeamertemplate{frametitle}[center]
\setbeamercolor{normal text}{bg=white, fg=black}
\setbeamertemplate{items}[circle]
\setbeamerfont{frametitle}{size=\huge}
\setbeamertemplate{navigation symbols}{} %no nav symbols
\setbeamertemplate{theorems}[normal font]

\usefonttheme{professionalfonts}
\usepackage[T1]{fontenc}

\newcommand{\slide}[1]{%
{
\usebackgroundtemplate{\includegraphics[width=\paperwidth]{pdfs/#1}}
\begin{frame}[plain]\end{frame}
}
}

\newcommand{\todoslide}[1]{%
\begin{frame}[plain]
	\centering\Large\textcolor{red}{TODO: #1}
\end{frame}
}


\usepackage[style=alphabetic]{biblatex}
\usepackage{lato} % FR: You might have to install this!


\AtBeginBibliography{\tiny}

\DeclareLabelalphaTemplate{
	\labelelement{
		\field[strwidth=3,strside=left,names=1,noalphaothers=true]{labelname}
	}
	\labelelement{
		\field[strwidth=2,strside=right]{year}
	}
	\labelelement{
		\literal{~}
	}
	\labelelement{
		\field{shorttitle}
	}   
}
\addbibresource{biblio.bib}


% \DeclareLabelalphaTemplate{
% 	\labelelement{
% 		\field[varwidthnorm]{labelname}
% 	}
% 	\labelelement{
% 		\literal{--}
% 	}
% }

\begin{document}

% Keynote. Aim for 40 minutes (55 minutes for entire thing, including questions)

%https://arxiv.org/abs/2112.10562 Our paper (version on Slack supercedes this; in press at EuJC - I'm supposed to be reviewing proofs this week)

% Abstract. 
%Over the last ten years, there has been significant work on adapting tools and techniques from structural graph algorithms for practical network analysis. One promising direction has arisen from the notion of bounded expansion, a broad notion of sparsity introduced by Nešetřil and Ossona de Mendez in 2012. Bounded expansion seems to exhibit a Goldilocks-like property of both encompassing many classes of real-world networks while still admitting (theoretically) efficient parameterized algorithms for many graph optimization problems. Taken together, this suggests that it may be a rich source of efficient algorithms for analyzing real-world networks. Further, bounded expansion includes many well-studied classes (e.g. bounded degree, H-minor-free, bounded crossing number) and can be characterized by – and thus relates – a plethora of independently interesting parameters (e.g. density of shallow minors, uniform quasi-wideness, and generalized coloring numbers). Many FPT algorithms take advantage of the inherent bounds on these parameters in bounded expansion classes in order to achieve polynomial runtimes. However, in order for these approaches to be practically useful, we must understand the computability of these parameters. Existing work implicitly relies on "good enough" approximations to claim the algorithms achieve fixed-parameter tractability – for example, a -approximation for -centered colorings. Unfortunately, the magnitude of the approximation factors inherently renders the approaches infeasible on real-world instances. In contrast, other parameters characterizing bounded expansion such as the generalized coloring numbers are not even known to be NP-hard to compute exactly. In this talk, we provide an overview of this space, including new results with Breen-McKay and Lavallee on the hardness and approximability of generalized coloring numbers – parameters of independent interest which were introduced by Kierstead and Yang in 2003. We also highlight several remaining open problems where affirmative answers could move the field significantly closer to achieving its practical potential.


%Title 
%TODO: Beautify slide. 
% Idea: a diamond is hard, and can act as a prism which %creates a rainbow -- maybe having a platypus peering through one, which is casting multicolor light onto a graph? Low priority, all things considered. 
%BS: There's a diamond in the images folder now.
\slide{title}

% Acknowledgements 
\slide{thanks}


% [10 mins] on FPT, structural graph theory -> bounded expansion
% [20 mins] coloring numbers, connections to various areas (algorithms, understand relationship to other parameters, combinatoric interest)
%   -- history: games on graphs (Kierstad & Yang)
%   -- highlight connections
%   -- relationship to other paramets
% 		> Admissibility?
% 		> 'Limits' treedepth/treewidth
%   	> Cops & Robbers! 
% [10 mins] new results
%   -- hard
%	-- greedy approximation 
% [3-5 mins] open questions, at least one per area 
%		> algorithms: constant factor approx., e.g. independent of wcol_r even 
%		  for just r=2)
% 		> relationships: ?
% 		> combinatorics: bounds for some graph classes

% Reference slide!
%  -- References as [Name, Year, Short venue code]

%Section Header 
%TODO: Consider another image - I don't understand sculpture analogy here.
\slide{title-sparsity}


%% Wide gap between theoretical results and real-world data analysis 
\slide{theory-vs-practice}
\slide{many-worlds}

%Parameterized Complexity - quick recap of definitions/motivation
\slide{fpt}

% -- TODO:FR: Replace BE by crossed-out shallow minor
\slide{hierarchy-labeled}

%Idea: 
% Lots of interest in using structure and sparsity to design efficient algorithms, but also lots of evidence that existing parameters are typically too big in realistic networks. 
%BDS: Didn't quite get to making slide outline(s) for this, but did organize the content a bit. Leaning towards keeping it on the short side - I have a lot of material to fit in. Do you think it works on one slide? Or do we need two? 

%TODO:BDS: Add citations, update titles
\slide{unreal-structure}
\slide{unreal-structure-b}
%Treewidth 
%This paper argues that treewidth is super useful in biology, but networks have prohibitively high treewidth.
%https://almob.biomedcentral.com/articles/10.1186/s13015-022-00213-z might have some pointers? Also I need to read this! 
% FR: Treewidth of database networks (https://arxiv.org/pdf/1901.06862.pdf) and https://ieeexplore.ieee.org/document/6729484 (or Internet Math version) on social networks 

%Planarity 
% FR: Cute paper on planar representations of (non-planar) street networks, 
%     this one is a bit more social-sciency (https://journals.sagepub.com/doi/pdf/10.1177/2399808318802941?casa_token=AYfYJzbcxRIAAAAA:Hoi86T72xpKBlK_ntr-5pQRhxzQ6hzXrM8rUiFjYKy_FY1soOAWwSpQbqKZ9hpv12UWRbvSWn5I)
% FR: Planarization of biological networks (https://ieeexplore.ieee.org/stamp/stamp.jsp?arnumber=9669815&casa_token=Q7pDQ0KMis4AAAAA:p7lFIY2FZ3wiJVpzIOAMLFaDS2SaJ97du4Q1vRfRaT75KozwMlizUXzf8Zrf94mC1LZYnpqyOIc&tag=1)


% BS: Not sure these fit in the current pitch?
%Near-Cliques in c-closed classes (Jain et al) - this one is a bit more of a stretch, and not on the hierarchy; on the other hand, Shweta is my postdoc and on the job market (in CS), so.... 
%https://arxiv.org/abs/2007.09768
% Pal, Irene and Felix: "Harmless Set" which is inspired by networks questions https://arxiv.org/abs/2111.11834

% TODO:FR: Replace BE by crossed-out shallow minor
% TODO:FR: Highlight Bounded Expansion & put it in the "Goldilocks Zone" 
\slide{hierarchy-sparse}

%Recent work has shown that bounded expansion may be in the ``goldilocks'' zone  - define in terms of shallow minors, mention Nesetril/Ossona de Mendez, show connection to other well-known structural graph classes via hierarchy, mention connections to uniform quasi-wideness, p-centered colorings. 
\slide{be-shallow}
\slide{be-shallow-A}
\slide{be-shallow-B}

%Evidence that BE exists in random graph models 
\slide{random-models}

%Algorithmically useful, lots of characterizations
\slide{be-overview}
\slide{be-applications}

%Steps towards Practice:

% FR: This could be an example of a wcol-based algorithm. Important here:
%     computing a good wcol ordering is the first and maybe most important step
%     for a fast algorithm
%TODO: replace Magic Hammer by something more fitting? (idea: rainbow??)
%Algorithmic pipeline is based on computing an ordering of V(G) with special properties 
\slide{wcol-counting}

%TODO:BDS: Mention IBD work? 
\slide{spacegraphcats}

%Section header 
\slide{title-coloring-numbers}

%Definition of generalized coloring numbers 
%Strong
\slide{be-col}
\slide{be-col-A}
\slide{be-col-B}
%Weak 
\slide{be-wcol}
\slide{be-wcol-A}
\slide{be-wcol-B}
\slide{be-wcol-col}

%TODO: decide if degeneracy definition is needed.
\slide{origin-story}
\slide{origin-story-A}
\slide{origin-story-B}

%\slide{degeneracy}

%Connections with game chromatic number 
\slide{games}
\slide{games-A}


% TODO:BDS: Add text from Sketch 
% Some recent work on generalized coloring numbers in special classes 
% https://www.sciencedirect.com/science/article/pii/S0012365X21003447 (Almuhim, Kierstead 2022 Discrete Math; may not include this, but weak-2 on planar has been a big push: best known is 23 now -> star list chromatic number; conjecture is 18)
%https://arxiv.org/abs/2201.09340 (Nederlov, Pilipczuk, Wegrzycki - planar via Koebe orderings/coin models ArXiv 2022), 
%https://www.sciencedirect.com/science/article/abs/pii/S0012365X19303905
% https://arxiv.org/abs/1602.09052 (van den Heuvel et al; excluding a fixed minor - linear and polynomial bounds including colr(G)≤5r+1  for planar; EuJC 2017) https://www.sciencedirect.com/science/article/pii/S0195669817300938
%https://arxiv.org/abs/2102.10061 (Joret & Micek; wcol only - treewidth Theta(r^(k−1) log r)and planar; Electronic Journal of Combinatorics 2022) https://www.combinatorics.org/ojs/index.php/eljc/article/view/v29i1p60
%https://arxiv.org/abs/1907.10962 (Kierstead, Yang, Yi, graph powers, Discrete Math 2020)
\slide{planarity}


\slide{uniform-orderings}
\slide{uniform-orderings-A}
% Uniform Orderings for multiple radii simultaneously
%https://arxiv.org/abs/1907.12149
 %https://www.sciencedirect.com/science/article/abs/pii/S0195669820301359 
 % van den Heuvel, Kierstead, EuJC 2021


%TODO: Beautify
\slide{twinwidth}
%https://arxiv.org/abs/2104.09360 
%https://www.sciencedirect.com/science/article/abs/pii/S0012365X21004593
%Dreier et al Discrete Math 2022
%Observe (out loud) that recently work addresses many structural and algorithmic questions such as FPT model checking, graph enumeration, graph coloring, matrices and ordered graphs, and transductions of permutations.
%Similar exponential upper bounds for wcol and adm. 


%https://arxiv.org/abs/2302.00352
%TODO:BDS: make sure you understand the rules of the game.
%Cop-Width Game: robber runs at speed r, for some fixed r ∈ N ∪ {∞}. That is, in each round, after the cops have taken off in their helicopters to their new positions ***which are known to the robber***, and before the helicopters have landed, the robber may traverse a path of length at most r that does not run through a cop that remains on the ground (he may also stay put). We call this game the cop-width game with radius r and width k, if there are k cops, and the robber can run at speed r. Obs: copwidth1(G) = degeneracy(G) + 1. (note to self: minimax thm comes from fact that every graph G is either d-degenerate, or it has a subgraph H in which every vertex has degree larger than d.)
% There is a variant with exact min-max theorem (cops announce moves first)
% Leads to admr (G) + 1 leq  copwidthr′ (G) leq  scolr (G) + 1
% Basically also studied in Richerby & Thilikos WG 2008 and Limnios et al TCS 2020 but without connections to coloring numbers 
\slide{cops-and-robbers}
\slide{cops-and-robbers-A}
\slide{cops-and-robbers-B}
\slide{copwidth}
\slide{copwidth-A}
\slide{copwidth-B}
\slide{copwidth-C}


% Motivated from cops & robbers
% -- "Interpolate" between degeneracy and td/tw
% -- Degeneracy as "practical" parameter (netsci), td/tw as "theoretical" - should we add the planets here?
%TODO:FR: Can we annotate this with known hardness (maybe on a reveal slide)? 
\slide{col-wcol-td-tw} 

\slide{title-algorithmic}

% BS: note that this gives the cops strategy in the copwidth1 case (v + its d back-nbrs)
\slide{degeneracy-computation}
\slide{degeneracy-computation-A}
\slide{degeneracy-computation-B}

%TODO:FR: Add property about number of back-neighbors. 
\slide{degeneracy-computation-C}

\slide{hardness}

%T-DO:??: Are these the right bounds for Dvorak??? We should have improved his approximation algorithm.
% FR: Fixed, they are incorrect in you paper though (unless I was too tired).
% TODO: Add citations. Note that wcol_r bound is from a more recent paper (dvovrak2022weighted), it was published on arxiv in '21
\slide{order-approx}

% If you could compute admissibility (r-path packing), greedy would work here like in degeneracy, unfortunately this is not computationally feasible. 
\slide{order-algo-estimate}
%We use estimated backconnectivity -- restrict to shortest r-qualifying paths.
%Mention that we get Catalan numbers?

%TODO:BDS: Add open questions, and images for three categories (algorithms, bounds on classes, ??)
\slide{open-questions} 
% Any Constant-factor approximation (get rid of the k)
% FR: We don't have any bounds/algorithm specifically for _networks_
% Are game coloring numbers algorithmicall useful? (only known result is for a Bollobas-Eldridge type packing problem via Kierstead and Kostochka)
% What is the asymptotics of the maximum of wcolr(G) when G is planar? It
%is known to be Ω(r2 log r) and O(r3). Joret et al conjecture Θ(r^2 log r); Pilipczuk et al think maybe via Koebe orderings - is it true  wcold(G, 􏰒) \leq  d^3 · ln^O(1) d for every Koebe order?

%BS: I saw this in your deck and almost asked for it :) 
\slide{thanks-end}

%Advertise for ACDA23? 
%This is a slide I can make if I decide I want it.

% Bibliography
\setbeamercolor{background canvas}{bg=talk_bg}
\begin{frame}[allowframebreaks]
	\nocite{flipwidth}
	\nocite{tree-like}
	\nocite{treewidth-db}
	\nocite{tree-diet}
	\nocite{planar-street}
	\nocite{planar-bio}
	\nocite{ColoringHardness}
	\nocite{Joret2022}
	\nocite{Brown2020}
	\nocite{Nesetril2012}
	\nocite{Nederlof2022}
	\nocite{Reiter2022}
	\nocite{vandenHeuvel2021}
	\nocite{vandenHeuvel2017}
	\nocite{Demaine2019}
	\nocite{dvovrak2022weighted}
	% TODO:BDS: add all bib entries 
	\printbibliography
\end{frame}

\end{document}

\documentclass[xcolor=table]{beamer}
%\usepackage{etex} % Fixes out-of-registers error. Probably some bad package.

\usepackage[utf8x]{inputenc}
%\usepackage{default}
\usepackage[pdftex]{epsfig}
\usepackage[table]{xcolor}

\usepackage{tikz}
\usetikzlibrary{calc}

\usepackage{pdfpages} % Include other pdf files as slides

\setbeamertemplate{navigation symbols}{}

\renewcommand\sfdefault{phv}
\renewcommand\familydefault{\sfdefault}
\usetheme{default}
\usepackage{color}
\useoutertheme{default}

\setbeamertemplate{frametitle}[center]
\setbeamercolor{normal text}{bg=white, fg=black}
\setbeamertemplate{items}[circle]
\setbeamerfont{frametitle}{size=\huge}
\setbeamertemplate{navigation symbols}{} %no nav symbols
\setbeamertemplate{theorems}[normal font]

\usefonttheme{professionalfonts}
\usepackage[T1]{fontenc}

\newcommand{\slide}[1]{%
{
\usebackgroundtemplate{\includegraphics[width=\paperwidth]{pdfs/#1}}
\begin{frame}[plain]\end{frame}
}
}

\newcommand{\todoslide}[1]{%
\begin{frame}[plain]
	\centering\Large\textcolor{red}{TODO: #1}
\end{frame}
}


\usepackage[style=alphabetic]{biblatex}
\AtBeginBibliography{\footnotesize}

\DeclareLabelalphaTemplate{
	\labelelement{
		\field[strwidth=3,strside=left,names=1,noalphaothers=true]{labelname}
	}
	\labelelement{
		\field[strwidth=2,strside=right]{year}
	}
	\labelelement{
		\literal{~}
	}
	\labelelement{
		\field{shorttitle}
	}   
}
\addbibresource{biblio.bib}


% \DeclareLabelalphaTemplate{
% 	\labelelement{
% 		\field[varwidthnorm]{labelname}
% 	}
% 	\labelelement{
% 		\literal{--}
% 	}
% }

\begin{document}

% Keynote. Aim for 40 minutes (55 minutes for entire thing, including questions)

%https://arxiv.org/abs/2112.10562 Our paper (version on Slack supercedes this; in press at EuJC - I'm supposed to be reviewing proofs this week)

% Abstract. 
%Over the last ten years, there has been significant work on adapting tools and techniques from structural graph algorithms for practical network analysis. One promising direction has arisen from the notion of bounded expansion, a broad notion of sparsity introduced by Nešetřil and Ossona de Mendez in 2012. Bounded expansion seems to exhibit a Goldilocks-like property of both encompassing many classes of real-world networks while still admitting (theoretically) efficient parameterized algorithms for many graph optimization problems. Taken together, this suggests that it may be a rich source of efficient algorithms for analyzing real-world networks. Further, bounded expansion includes many well-studied classes (e.g. bounded degree, H-minor-free, bounded crossing number) and can be characterized by – and thus relates – a plethora of independently interesting parameters (e.g. density of shallow minors, uniform quasi-wideness, and generalized coloring numbers). Many FPT algorithms take advantage of the inherent bounds on these parameters in bounded expansion classes in order to achieve polynomial runtimes. However, in order for these approaches to be practically useful, we must understand the computability of these parameters. Existing work implicitly relies on "good enough" approximations to claim the algorithms achieve fixed-parameter tractability – for example, a -approximation for -centered colorings. Unfortunately, the magnitude of the approximation factors inherently renders the approaches infeasible on real-world instances. In contrast, other parameters characterizing bounded expansion such as the generalized coloring numbers are not even known to be NP-hard to compute exactly. In this talk, we provide an overview of this space, including new results with Breen-McKay and Lavallee on the hardness and approximability of generalized coloring numbers – parameters of independent interest which were introduced by Kierstead and Yang in 2003. We also highlight several remaining open problems where affirmative answers could move the field significantly closer to achieving its practical potential.

%Title - needs updating to match style (whatever that ends up being), have correct title, event, and date
%TODO: One idea I had about "hard look at generalized coloring numbers" was that a diamond is hard, and can act as a prism which 
%creates a rainbow -- maybe having one casting multicolor light onto a graph would be a neat title image? Low priority, all things considered. 
\slide{title}




% [10 mins] on FPT, structural graph theory
% [20 mins] coloring numbers, connections to various areas (algorithms, understand relationship to other parameters, combinatoric interest)
%   -- history: games on graphs (Kierstad & Yang)
%   -- highlight connections
%   -- relationship to other paramets
% 		> Admissibility?
% 		> 'Limits' treedepth/treewidth
%   	> Cops & Robbers! 
% [10 mins] new results
%   -- hard
%	-- greedy approximation 
% [3-5 mins] open questions, at least one per area 
%		> algorithms: constant factor approx., e.g. independent of wcol_r even 
%		  for just r=2)
% 		> relationships: ?
% 		> combinatorics: bounds for some graph classes

% Reference slide!
%  -- References as [Name, Year, Short venue code]

\todoslide{Section: Sparse graph theory}


%Bringing structural graph algorithms to real-world data analysis

%% Wide gap between theoretical result and real-world data analysis
% Planets analogy - need to add microbes to bioinformatics

%Parameterized Complexity - quick recap of definitions/motivation
% TODO -- Let's use the SLC Whale 'Out of the Blue' as the new sculpture! It's really controversial here, and a great talking point.
\slide{fpt}

%Lots of recent work, much of which is focused on sparsity of networks -- would be good to mention some work that's not bounded expansion and not associated with me here.  
%https://arxiv.org/abs/2007.09768
%https://almob.biomedcentral.com/articles/10.1186/s13015-022-00213-z might have some pointers? Also I need to read this! 
% FR: Treewidth of real-world networks (https://arxiv.org/pdf/1901.06862.pdf) and https://ieeexplore.ieee.org/document/6729484 

% FR: Cute paper on planar representations of (non-planar) street networks, 
%     this one is a bit more social-sciency (https://journals.sagepub.com/doi/pdf/10.1177/2399808318802941?casa_token=AYfYJzbcxRIAAAAA:Hoi86T72xpKBlK_ntr-5pQRhxzQ6hzXrM8rUiFjYKy_FY1soOAWwSpQbqKZ9hpv12UWRbvSWn5I)
% FR: Planarization of biological networks (https://ieeexplore.ieee.org/stamp/stamp.jsp?arnumber=9669815&casa_token=Q7pDQ0KMis4AAAAA:p7lFIY2FZ3wiJVpzIOAMLFaDS2SaJ97du4Q1vRfRaT75KozwMlizUXzf8Zrf94mC1LZYnpqyOIc&tag=1)
% FR: Pal, Irene and I did some work on "Harmless Set" which is inspired by networks questiosn https://arxiv.org/abs/2111.11834


%One key notion is bounded expansion - define in terms of shallow minors, mention Nesetril/Ossona de Mendez, show connection to other well-known structural graph classes via hierarchy, mention connections to uniform quasi-wideness, p-centered colorings. 
\slide{be-shallow}
\slide{be-shallow-A}
\slide{be-shallow-B}

% -- TODO: Replace BE by crossed-out shallow minor
\slide{hierarchy-sparse}
% -- Maybe something about BE being the "Goldilocks zone"? Yes! call-back to planets analogy in motivation section.

% Need to cite https://www.sciencedirect.com/science/article/pii/S0022000019300418
\slide{random-models}

%What can you do in bounded expansion classes? Mention theoretical results + Mandoline, SpaceGraphCats, etc. 
\slide{be-overview}
\slide{be-applications}

% FR: This could be an example of a wcol-based algorithm. Important here:
%     computing a good wcol ordering is the first and maybe most important step
%     for a fast algorithm
% FR: We probably want to replace the "Magic hammer" by something more fitting 

%Algorithmic pipeline is based on computing an ordering of V(G) with special properties 
\slide{wcol-counting}


\todoslide{Section: Generalized Coloring Numbers - what they are and why you should care, beyond BE}
%Definition of generalized coloring numbers 

\slide{be-col}
\slide{be-col-A}
\slide{be-col-B}

\slide{be-wcol}
\slide{be-wcol-A}
\slide{be-wcol-B}
%scolr(G) ≤ wcolr(G) ≤ (scolr(G))r.

% See SlideSketches for distillation and imagery suggestions on all of these. 

%TODO: decide if degeneracy definition is needed.
\todoslide{Origin Story}
%Originally introduced by Kierstead & Yang (Order 2003)
%Generalizes coloring number (relates to chromatic number, list coloring)
%Connections to game chromatic number

% For a graph G, σ ∈ Π and x ∈ V , let col(G,σ,x) be one more than the number of neighbors y ∈ NG(x) with y <σ x. %The coloring number col(G) of G is the least integer s such that every subgraph of G contains a vertex of degree less than s. 
%The coloring number of G, denoted col(G), is defined by
%col(G) = min max col(G, σ, x). σ∈Π x∈V
%In recent terminology, the coloring number of a graph is one more than its degeneracy; under an older definition of degeneracy they were the same. Greedily coloring the vertices of G in an ordering that witnesses its coloring number, shows that
%χ(G) ≤ ch(G) ≤ col(G),
%where χ(G) and ch(G) denote the chromatic and list chromatic number of G, respectively.

%Do I need degeneracy here? 
%\slide{degeneracy}

% If the vertices of G are colored greedily so that no vertex v receives the same color as any other vertex in S2[G,σ,v], then the resulting coloring is an acyclic coloring, so
%cha(G) ≤ scol2(G),
%If the vertices of G are colored greedily so that no vertex v receives the same color as any vertex in W2[G,σ,v], then the resulting coloring is a star coloring, so chs(G) ≤ wcol2(G),

\todoslide{It's all fun and games...}
%The r-ordering game is played on a graph G by two players, Alice and Bob. The game lasts for n = |G| turns. The players take turns choosing unchosen vertices with Alice playing first until there are no unchosen vertices left. This creates an ordering σ ∈ Π(G) of G, where vi is the vertex chosen at the i-th turn and v1 <σ v2 <σ ··· <σ vn. The score of the game is scolr(G,σ). Alice’s goal is to minimize the score while Bob’s goal is to maximize the score. The game r-coloring number of G, denoted gcolr(G), is the least s such that Alice can always achieve a score of at most s, regardless of how Bob plays.
%All graphs G satisfy gcolr(G) ≤ 3(wcol2r(G))^2 ≤ 3(scol2r(G))^{4r} for all r.

\todoslide{To Planarity and Beyond!}
% https://www.sciencedirect.com/science/article/pii/S0012365X21003447 (Almuhim, Kierstead 2022 Discrete Math; may not include this, but weak-2 on planar has been a big push: best known is 23 now -> star list chromatic number; conjecture is 18)
%https://arxiv.org/abs/2201.09340 (Nederlov, Pilipczuk, Wegrzycki - planar via Koebe orderings/coin models ArXiv 2022), 
%https://www.sciencedirect.com/science/article/abs/pii/S0012365X19303905
% https://arxiv.org/abs/1602.09052 (van den Heuvel et al; excluding a fixed minor - linear and polynomial bounds including colr(G)≤5r+1  for planar; EuJC 2017) https://www.sciencedirect.com/science/article/pii/S0195669817300938
%https://arxiv.org/abs/2102.10061 (Joret & Micek; wcol only - treewidth Theta(r^(k−1) log r)and planar; Electronic Journal of Combinatorics 2022) https://www.combinatorics.org/ojs/index.php/eljc/article/view/v29i1p60
%https://arxiv.org/abs/1907.10962 (Kierstead, Yang, Yi, graph powers, Discrete Math 2020)


\todoslide{One Ring to Rule Them}
% Uniform Orderings 
%https://arxiv.org/abs/1907.12149
 %https://www.sciencedirect.com/science/article/abs/pii/S0195669820301359 
 % van den Heuvel, Kierstead, EuJC 2021

% An obvious question concerning generalized coloring numbers is whether an ordering that is “good” for one distance r is also “good” for a different distance r′. 
%this need not be the case: in Example 2.1 we will show that for all r, r′ ∈ N with r ̸= r′, there exists a graph G such that for all σ ∈ Π(G) either scolr(G) < scolr(G, σ) or scolr′ (G) < scolr′ (G, σ).
%Question of Dvorak: Is it true that for all functions c : N → N, there exists a function c∗ : N → N, such that for every graph G with scol at most c(r), there exists an ordering σ∗ ∈ Π(G) such that scolr(G,σ∗) ≤ c∗(r) for all r ∈ N? (Kreutzer et al conjectured yes)
% Main result: For any graph G, there exists an ordering σ∗ of G such that for all r ∈ N we have scolr(G, σ∗) ≤ (2r + 1)(scol2r(G))^4r.

% Not planning to include: Connections to posets 
% Streib and Trotter [24] proved that every poset whose cover graph is planar, has dimension bounded by a function of its height. Then Joret et al. [9] used generalized coloring numbers to prove that every monotone graph class G is nowhere dense if and only if for every integer h ≥ 1 and real number epsilon > 0, every n-element poset of height at most h whose cover graph is in G has dimension O(n^epsilon).

\todoslide{Twinwidth (twice the fun, half the sleep)}
%https://arxiv.org/abs/2104.09360 
%https://www.sciencedirect.com/science/article/abs/pii/S0012365X21004593
%Dreier et al Discrete Math 2022
%twin-width graph parameter, defined by Bonnet, Kim, Thomassé and Watrigant FOCS 2020 as a generalization of a width invariant for classes of permutations defined by Guillemot and Marx SODA 2014. Recently work on structural and algorithmic questions such as FPT model checking [5], graph enumeration, graph coloring, matrices and ordered graphs, and transductions of permutations.
% Known: graph class with bounded twin-width excludes some biclique as a subgraph if and only if it has bounded expansion
% Also known (Bonnet et al SODA 2021): For every integer r there exists a function fr : N × N → N such that if G is a graph with twin-width t and no Ks,s-subgraph, then we have wcolr(G) ≤ fr(t,s). Similar bounds also exist for scolr and admr. No indication of how to compute such binding functions.
% Main result: prove that a graph G with no Ks,s-subgraph and twin-width d has admr, scolr and wcolr bounded from above by an exponential function of r, and that we can construct graphs achieving such a dependency in r. In particular, scolr(G) ≤ (d^r + 3)s (Theorem 2). On the other hand, one can choose G such that scolr(G) ≥ (d−4)^rs 

\todoslide{Cops and Robbers}
%Recent work characterizing coloring numbers using Cops & Robbers-style game 
%https://arxiv.org/abs/2302.00352
% Toruńczyk ArXiv 2023 
% Original game: The robber stands on a vertex of the graph, and can at any time run at great speed to any other vertex along a path of the graph. He is not permitted to run through a cop, however. There are k cops, each of whom at any time either stands on a vertex or is in a helicopter (that is, is temporarily removed from the game). The objective of the player controlling the movement of the cops is to land a cop via helicopters on the vertex occupied by the robber, and the robber’s objective is to elude capture. (The point of the helicopters is that cops are not constrained to move along paths of the graph – they move from vertex to vertex arbitrarily.) The robber can see the helicopter approaching its landing spot and may run to a new vertex before the helicopter actually lands. 
%Original result: the least number of cops needed to catch a robber on a graph G is equal to one plus the treewidth of G. min-max theorem: either the cops have a winning strategy of a particularly simple, monotone form, which can be described by a tree decomposition of the graph, or otherwise, the robber has a winning strategy of a particularly simple form, called a haven.

%Cop-Width Game: robber runs at speed r, for some fixed r ∈ N ∪ {∞}. That is, in each round, after the cops have taken off in their helicopters to their new positions, which are known to the robber, and before the helicopters have landed, the robber may traverse a path of length at most r that does not run through a cop that remains on the ground (he may also stay put). We call this game the cop-width game with radius r and width k, if there are k cops, and the robber can run at speed r. Obs: copwidth1(G) = degeneracy(G) + 1. (note to self: minimax thm comes from fact that every graph G is either d-degenerate, or it has a subgraph H in which every vertex has degree larger than d.)
%Main Result: admr (G) + 1 leq copwidthr (G) leq wcol2r (G) + 1.
% There is a variant with exact min-max theorem (cops announce moves first)
% Leads to admr (G) + 1 leq  copwidthr′ (G) leq  scolr (G) + 1
% Basically also studied in Richerby & Thilikos WG 2008 and Limnios et al TCS 2020 but without connections to coloring numbers 

% Motivated from cops & robbers
% -- "Interpolate" between degeneracy and td/tw
% -- Degeneracy as "practical" parameter (netsci), td/tw as "theoretical"
\slide{col-wcol-td-tw} 

\todoslide{Section: Algorithmic questions}

% FR: Degeneracy algorithm somewhere here (animation)
% BS: note that this gives the cops strategy in the copwidth1 case (v + its d back-nbrs)
\slide{degeneracy-computation}
\slide{degeneracy-computation-A}
\slide{degeneracy-computation-B}
\slide{degeneracy-computation-C}

%But are they computable? Until recently this was open. 

% Cite Breen-McKay, Lavallee, Sullivan, EuJC 2023 (In Press)
%Negative result: NP-hard to compute (weak and strong, all radii)

%End of the road? No - an approximation is good enough! 

%Dvorak's approximation.
% -- Just the approx. bounds (weak and strong)

%New greedy algorithm based on admissibility gives improved approximation. This should be several slides.

% Definition of Admissibility 
\slide{be-adm}
\slide{be-adm-A}
\slide{be-adm-B} 

\todoslide{Open Problems}
%What's still open? 
% Any Constant-factor approximation (get rid of the k)
% FR: We don't have any bounds/algorithm specifically for _networks_
% Are game coloring numbers algorithmicall useful? (only known result is for a Bollobas-Eldridge type packing problem via Kierstead and Kostochka)
% What is the asymptotics of the maximum of wcolr(G) when G is planar? It
%is known to be Ω(r2 log r) and O(r3). Joret et al conjecture Θ(r^2 log r); Pilipczuk et al think maybe via Koebe orderings - is it true  wcold(G, 􏰒) \leq  d^3 · ln^O(1) d for every Koebe order?

% Acknowledgements - have to decide whether to do at beginning or end 
% TODO: Update this to match talk.
\slide{thanks}

%Advertise for ACDA23? 


% Bibliography
\setbeamercolor{background canvas}{bg=talk_bg}
\begin{frame}
	\cite{flipwidth}
	\printbibliography
\end{frame}

\end{document}
